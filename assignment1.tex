\documentclass[10pt]{article}
\usepackage{xfrac} %create slanted fraction

\usepackage{graphicx}
\usepackage{pgfplots}

\usepackage{amsfonts}
\usepackage{amsmath}
\usepackage{amssymb}
\usepackage{amsthm}
\usepackage{setspace}
\usepackage{indentfirst}
\usepackage[top=1in, bottom=1in, left=1.25in, right=1.25in]{geometry}
\linespread{1.2}

\theoremstyle{plain}
%\newtheorem{theorem}{Theorem}[section]
\newtheorem{theorem}{Theorem}
\newtheorem{proposition}[theorem]{Proposition}
\newtheorem{lemma}[theorem]{Lemma}
\newtheorem{corollary}[theorem]{Corollary}


\theoremstyle{definition}
\newtheorem{definition}[theorem]{Definition}
\newtheorem{example}[theorem]{Example}

\newtheorem{algorithm}[theorem]{Algorithm}


\theoremstyle{remark}
\newtheorem{remark}[theorem]{Remark}

%\numberwithin{equation}{section}



\title{Compulsory Assignment 1}
\author{CHU Kailu 201302193}
\date{\today}

\begin{document}
\maketitle

\section{Problem Description}
We are considering how we arrange our food for nutrition balance.
There are some basic constraints, 

\begin{enumerate}
\item The total \textbf{energy} should be at least 10000kJ;
\item The energy come from \textbf{fat} should be between 20\% and 30\% of the total energy;
\item The energy come from \textbf{carbohydrates} should be between 55\% and 60\% of the total energy;
\item Assume we get the remained energy from \textbf{protein}.
\item 1g fat(resp. carbohydrates and protein) contain 38kJ(17kJ and 17kJ resp. carbohydrates and protain);
\end{enumerate}

Now we should decide what we eat in one day and find out the energy disribution. 
I pick up the following food and list their energy and price,

\begin{center}\begin{tabular}{|l|l|l|l|l|}
\hline
Food & Fat & Carbohydrates & Protein & Price \\
\hline
\textbf{Skimmed Milk} & 0.3\% & 4.7\% & 3.5\% & 8Dkk/kg \\
\hline
\textbf{Raw Salmon} & 10.9\% & 0\% & 19.9\% & 146Dkk/kg \\
\hline
\textbf{Raw Cucumber} & 0.1\% & 2.1\% & 0.7\% & 40Dkk/kg \\
\hline
\textbf{Savoy Cabbage} & 0.1\% & 6.1\% & 2.0\% & 30Dkk/kg \\
\hline
\textbf{Rice} & 1.2\% & 79\% & 1.2\% & 10Dkk/kg\\
\hline 
\end{tabular}\end{center}


\section{Solution}
Let $\mathcal{I}$ be the food set and $\mathcal{J}$ be the nutrition set.
Let $a_{ij} :=$ the percentage that food $i$ contain nutrition $j$ with,
$p_i:=$ the price(Dkk/kg) of food $i$, 
$e_j:=$ the energy(kJ/g = 1000kJ/kg = kkJ/kg) of nutrition $j$,
for $i\in \mathcal{I},j\in \mathcal{J}$.

To come up with the \textbf{Linear Problem}, 
we let $w_i$ be the variables standing for the consumption(kg) of food $i$.
We want to minimize the total cost(DKK) in one day 
$\sum_{i\in\mathcal{I}}p_iw_i$.

We let $s_{ij}:=w_ia_{ij}e_j$ 
denote the energy(kkJ) obtained in nutrition $j$ from food $i$,
for $i\in\mathcal{I},j\in\mathcal{J}$.
We let $E:=\sum_{i\in\mathcal{I},j\in\mathcal{J}}s_{ij}$ 
denote the total energy(kkJ) one may have in one day. 
Similarly, we let $E_f:=\sum_{i\in\mathcal{I}}s_{i,1}$
and $E_c:=\sum_{i\in\mathcal{I}}s_{i,2}$
denote the fat energy(kkJ) and carbohydrates energy(kkJ) respectively one may have in one day.

Now we can translate the total energe constraint as $E\geq10$ 
where 10kkJ equals to 10000kJ,
the fat energy constraint as $E_f\geq0.2E$ and $E_f\leq0.3E$,
the carbohydrates energy constraint as $E_c\geq0.55E$ and $E_c\leq0.6E$.
Also, we need to constraint all the variables nonnegative.

We get approximate optimal total price at $71.53$DKK by consuming about $862.9$g Milk, $415.6$g Salmon and $395.4$g Rice.
But at this solution, we do not have to eat any vegetable. 


We can assume $w_3$ and $w_4$ at least $0.1$ both which indicates we must eat some vegetable.

Now we get the approximate optimal total price at $77.98$DKK by consuming about
$782.0$g Milk, $416.6$g Salmon, $100.0$g Cucumber, $100.0$g Cabbage, $390.0$g Rice.
\end{document}
