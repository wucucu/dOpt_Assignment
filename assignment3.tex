\documentclass[10pt]{article}
\usepackage{xfrac} %create slanted fraction

\usepackage{graphicx}
\usepackage{pgfplots}

\usepackage{amsfonts}
\usepackage{amsmath}
\usepackage{amssymb}
\usepackage{amsthm}
\usepackage{setspace}
\usepackage{indentfirst}
\usepackage[top=1in, bottom=1in, left=1.25in, right=1.25in]{geometry}
\linespread{1.2}

\theoremstyle{plain}
%\newtheorem{theorem}{Theorem}[section]
\newtheorem{theorem}{Theorem}
\newtheorem{proposition}[theorem]{Proposition}
\newtheorem{lemma}[theorem]{Lemma}
\newtheorem{corollary}[theorem]{Corollary}


\theoremstyle{definition}
\newtheorem{definition}[theorem]{Definition}
\newtheorem{example}[theorem]{Example}

\newtheorem{algorithm}[theorem]{Algorithm}


\theoremstyle{remark}
\newtheorem{remark}[theorem]{Remark}

%\numberwithin{equation}{section}



\title{Compulsory Assignment 3}
\author{CHU Kailu 201302193}
\date{\today}

\begin{document}
\maketitle

\section{Problem Description}
We consider a game league playing a tournament with $N$ teams, 
numbered by $1,2,\dots,N$.
Each game is played by two team. 
Winning team gets $2$ points, while losing team gets $0$. 
Both teams get $1$ point in case of a tie.

Assume we are in the middle of the tournament, i.e., 
we have a score list for each team.
Besides, we have a future match list.

We want to find out if it is possible for team $1$ to win the tournament,
i.e., team $1$'s point strictly larger than any other team.

\section{Preparation for ILP Modelling}
We assume that team $1$ wins all its future matches. 
Then we get the final score, denoted as $T$, for team $1$.

Denote by $L$ the list of remaining matches, i.e., 
future matches that team $1$ is not involved.
There is no need to consider the matches involving team $1$,
as all team $1$ will get $0$ points by our assumption.

Denote by $s_i$ the current score of team $i$ for $i=2,3,\dots,N$.

We introduce decision variables $x_{m,t} \in \{0,1,2\}$, 
standing for the score that team $t$ got in the match $m$
where $m \in L$ and $t$ is one of the two teams in $m$.
For convienince, we denote $t_1(m)$ and $t_2(m)$ as the two teams in $m$.
We can treat $t_1$ as a map from $L$ to $\{2,3,\dots,N\}$, 
and the same goes for $t_2$.

Then we have team $i$'s final score is 
$s_i+ \sum_{m: t_1(m)=i \vee t_2(m)=i} x_{m,i}$, for $i = 2,3,\dots,N$.

\section{Completion for the ILP}
To guarantee team $1$ to be the unique winner, its final score must be largest.
Then we get contraints
\[
  s_i+ \sum_{m: t_1(m)=i}x_{m,i} + \sum_{m: t_2(m)=i}x_{m,i} \leq T-1
\]
for $i = 2,3,\dots,N$.

It's easy to see for each match $m \in L$, we get contraint
\[
  x_{m,t_1(m)} + x_{m,t_2(m)} = 2
\]
as team $t_1(m)$ and team $t_2(m)$ will get total score of $2$
no matter what the result of the match will be.

To sum up, we model the ILP as following,
\begin{align*}
  i \in \{2,\dots,N\}: & s_i+ \sum_{m:t_1(m)=i}x_{m,i} +
  \sum_{m:t_2(m)=i}x_{m,i} \leq T-1 & \\
  m \in L: & x_{m,t_1(m)} + x_{m,t_2(m)} = 2 & \\
  m \in L: & x_{m,t_1(m)},x_{m,t_2(m)} \in \{0,1,2\} &
\end{align*}.

As we focus on the feasiblity of the problem,
there is no need to consider the "maximum" function
when we look at the standard form,
\begin{align*}
  i \in \{2,\dots,N\}: &
  \sum_{m:t_1(m)=i}x_{m,i}+\sum_{m:t_2(m)=i}x_{m,i}
  \leq T-1-s_i \\
  m \in L: &
  x_{m,t_1(m)} + x_{m,t_2(m)} \leq 2 \\
  m \in L: &
  -x_{m,t_1(m)} - x_{m,t_2(m)} \leq -2 \\
  m \in L: &
  x_{m,t_1(m)},x_{m,t_2(m)} \geq 0 \\
  m \in L: &
  x_{m,t_1(m)},x_{m,t_2(m)} \in \mathbb{Z}
\end{align*}.

Observe that the coefficient of any variable is
$1$, $0$, or $-1$. 
Hence the problem is totally unimodular.
Then the basic solutions of the problem consists of integers. 
The problem can be solved by simplex algorithm.

\end{document}
