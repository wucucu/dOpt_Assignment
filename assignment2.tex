\documentclass[10pt]{article}
\usepackage{xfrac} %create slanted fraction

\usepackage{graphicx}
\usepackage{pgfplots}

\usepackage{amsfonts}
\usepackage{amsmath}
\usepackage{amssymb}
\usepackage{amsthm}
\usepackage{setspace}
\usepackage{indentfirst}
\usepackage[top=1in, bottom=1in, left=1.25in, right=1.25in]{geometry}
\linespread{1.2}

\theoremstyle{plain}
%\newtheorem{theorem}{Theorem}[section]
\newtheorem{theorem}{Theorem}
\newtheorem{proposition}[theorem]{Proposition}
\newtheorem{lemma}[theorem]{Lemma}
\newtheorem{corollary}[theorem]{Corollary}


\theoremstyle{definition}
\newtheorem{definition}[theorem]{Definition}
\newtheorem{example}[theorem]{Example}

\newtheorem{algorithm}[theorem]{Algorithm}


\theoremstyle{remark}
\newtheorem{remark}[theorem]{Remark}

%\numberwithin{equation}{section}



\title{Compulsory Assignment 2}
\author{CHU Kailu 201302193}
\date{\today}

\begin{document}
\maketitle

\section{Problem description}
Alice and Bob play a game by choosing a integer between $1$ to $n$ at the same time, where $n$ is a given postive integer.

If they choose a same number, Bob will pay Alice $1$ dollar. 
Bob wiil get $1$ dollar from Alice. 
Nothing will happen otherwise.

Let Alice be row player and Bob be column player.
We shall look at the $n \times n$ payoff matrix in the next section,
where the row index $i$ stands for Alice choosing integer $i$,
the column index $j$ stands for Bob choosing integer $j$.

\newpage

\section{Solution}
Denote the payoff matrix by $A:=\{a_{ij}\}_{n\times n}$. 
$a_{ij}$ is the payoff that Bob will pay to Alice.
If $a_{ij}$ is negative, Bob will get $-a_{ij}$ from Alice.

Then we have that $a_{ij} = 1$ if $i=j$; 
$a_{ij} = -1$ if $j-i=1$; 
$a_{ij} = 0$ otherwise.

Now consider the case $n = 3$.

\noindent\textbf{1} 
The matrix will be 
$
\begin{pmatrix}
1 & -1 & 0 \\
0 & 1 & -1 \\
0 & 0 & 1 \\
\end{pmatrix} 
$.

\noindent\textbf{2}
The value of the game is $1/6$.

Notice there is no saddle point
(i.e., the entry that largest among its column and smallest among its row).
Moreover, there is no column(resp. row) can be dominated by another column(resp. row).

Denote the row player's mixed strategy by $p^t:=(p_1,p_2,p_3)$, 
where $p_1+p_2+p_3=1$.
Calculate $p^tA$. We get $(p_1,-p_1+p_2,-p_2+p_3)$.
We want to maximize the minimum of the three entries.

Introduce a new variable $v$. Then we want to solve the following LP:
max $v$, s.t. $v \leq p_1$, $v \leq -p_1+p_2$, $v \leq -p_2+p_3$ and $p_1+p_2+p_3=1$.

At the optimal solution, all of $=$ must hold,
i.e., we should solve a linear equation system with four unknown variavle. 
Then we have $(v,p_1,p_2,p_3) = (1/6,1/6,1/3,1/2)$.

\noindent\textbf{3}
The optimal mixed strategy for Alice is $(1/6,1/3,1/2)$.

\noindent\textbf{4}
The optimal mixed strategy for Bob is $(1/2,1/3,1/6)$.

Similarly, we denote the column player's mixed stratedy by $q^t:=(q_1,q_2,q_3)$,
where $q_1+q_2+q_3=1$.
Then we have $Aq = (q_1-q_2,q_2-q_3,q_3)$.

As we already know the value of the game is $1/6$, 
we have the entries of $Aq$ are all equal to the value $1/6$.
Hence we get the optimal mixed strategy for Bob as $(1/2,1/3,1/6)$.

\noindent\textbf{5}
When $n=4$, the value is $1/10$.

We can get this as soon as we solve the sub-question \textbf{6}.

\noindent\textbf{6}
In general, the value is $1/\sum_{i=1}{n}i=2/n(n+1)$.

Denote the value by $v$. 
By the discussion in sub-question \textbf{2},
the value will be find by solving the system
$V=p_1=-p_1+p_2=-p_2+p_3=\dots=-p_{n-1}+p_n$ and $p_1+p_2+\dots+p_n=1$.
\end{document}
